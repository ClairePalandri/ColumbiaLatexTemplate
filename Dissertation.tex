%Dissertation Template for Columbia University Ph.D. programs
%By Stephanie Lackner (2017), changed from version by Charles McNamara (2016)
%I changed some content, but it follows more or less the same structure as Charles McNamara's version and I also kept many of his helpful comments throughout the document.
%It's probably a good idea to review the university guidelines just so you know what you want your dissertation to look like. You can read about those guidelines at this site: http://gsas.columbia.edu/content/formatting-guidelines.
%Good luck writing your dissertation!


% This is the main document file for the dissertation. You should not include any of your actual chapters or other substantive writing in this file.

%First we have to set up the style and formatting of the pages.
\documentclass[letterpaper,12pt, oneside]{memoir} %The memoir class is great for longer works that use separate chapters. The Dissertation Office recommends 10-pt Arial or 12-pt Times New Roman. I use 12-pt for readability.


%------------------------------------
%Loading (some optional) Packages. These differ from Charles McNamara's version.
\usepackage{verbatim} % For long comments. Needed if you want to set the documents up so that you can easily compile individual tex files.
\usepackage{graphicx} %Allows including images
\usepackage[space]{grffile} %To allow your path to contain spaces (I had everything on my Google Drive)
\usepackage{booktabs} %For nice lines in tables
\usepackage{longtable} %For using tables that are longer than one page
\usepackage{float} %to have "stuff" figures, footnote etc where it should be
\usepackage{rotating} %For rotating things such as tables (e.g. sidewaystable environment)
\usepackage{enumitem} % To fix spacing of bullet point lists
\usepackage{appendix}
\setlist{noitemsep} %removes space between items
%\usepackage{afterpage} %might be useful if you run into spacing problems
%\usepackage[labelfont={bf},font={small}]{caption} %Adjusting table and figure captions the way I prefer them
\usepackage[hidelinks, plainpages=false, pdfpagelabels]{hyperref} % Use if you want hyperlinks (e.g. in your table of contents). If you don't want them to be blue use the option "hidelinks".
%hperref sometimes has compatibility issues with other packages, which can often be solved by loading hyperref last. Otherwise more warnings appear about destinations with same identifier.

\newcommand{\totalN}{10,000 } %An important number that you can use in the text

\graphicspath{{"Figures/"}} %You can set a path where all the Figures are saved


%------------------------------------
%Here is some stuff on the bibliography. You want to keep your bibliography file in the same directory as this file.
%I use a different referencing approach than Charles McNamara. You might want to change this to whatever you want to do.
\usepackage[super]{natbib} %Nature citation style with superscripts
\setcitestyle{square} %And square brackets around the citation numbers
\setlength{\bibitemsep}{\baselineskip} %Skip lines between bibliography entries. Columbia requires that you skip a line between entries.


%------------------------------------
%Here you can set margins and other page formatting
\setlrmarginsandblock{3cm}{3cm}{*} %Left and right margin -- the dissertation office requires at least 1-inch margins (2.54 cm)
\setulmarginsandblock{3cm}{3cm}{*}  %Upper and lower margin -- same thing, at least 1-inch margins
\checkandfixthelayout %A function of the memoir class that finds the right number of lines per page and apparently tidies up the formatting in other mysterious ways...?


%------------------------------------
%Options that Charles McNamara used, but weren't useful for me
%\sloppy %If I don't include sloppy, then Greek and Latin words screw up margins all over. If you don't include weird languages in your dissertation, you can probably leave this one out.
%\chapterstyle{chappell} % Nice formatting for chapter headings. Check out the documentation for the memoir class for other options.
%\renewcommand*{\chapnumfont}{\normalfont\HUGE\bfseries\sffamily}
%\renewcommand*{\chaptitlefont}{\normalfont\HUGE\bfseries\sffamily}


%------------------------------------
%Adjusting all kinds of Spacing
%You can't use the parskip package in the memoir class, but the following commands can help instead.
\abnormalparskip{0pt}
\setlength\beforechapskip{0pt} %Adjust Chapter Title position
\footnotesep\baselineskip % Footnotes need to have a space between each one for Columbia's Dissertation Office.
\DoubleSpacing %Set roomier body text throughout your writing. The dissertation office requires that you use double-spacing throughout your main body text.
\expandafter\def\expandafter\quote\expandafter{\quote\SingleSpace} %Keep all block quotes single-spaced regardless of body text spacing.
\pagestyle{plain} %Put page numbers at the bottom-center for the whole dissertation. Columbia's dissertation office requires that the numbers appear at this location on the page throughout the document.

%\raggedbottom %If I didn't include this, the last page of each chapter would stretch the text out to fill the entire page.


%------------------------------------
% Below we start to set up the document itself.
\begin{document}


%------------------------------------
%The section that follows renders the "Cover pages and Abstract" part of your dissertation.

%Without this, you will receive a duplicate page identifier warning (hyperref package).
%It has to be combined with setting the pageanchor back to true after the abstract
\hypersetup{pageanchor=false}

%% Here's the Title Page
% This is the title page of the dissertation.
% You may need to change the text on this page for your particular department. Consult the Dissertation Office Formatting Guidelines.
\thispagestyle{empty} % No page number
\setcounter{page}{0} %For optimizing page number display in pdf reader
\begin{center}
  \SingleSpace

  \vspace*{2in}
  \Large
  [Title of your dissertation]

  \vspace*{0.3 in} % Get space between title and author name
  \normalsize
  [Firstname Lastname]

  \vspace{3.5in}

  Submitted in partial fulfillment of the\\
  requirements for the degree of\\
  Doctor of Philosophy\\
  in the Graduate School of Arts and Sciences

  \vfill

  COLUMBIA UNIVERSITY

  \bigskip % Get space between Columbia and the year

  2017
\end{center}


%% Here's the copyright page.
\include{./FrontMatter/Copyright}

%% Here's the Abstract
% This is the abstract of my dissertation.

\thispagestyle{empty} % No page number in entire abstract
\setcounter{page}{0} %For optimizing page number display in pdf reader
\vspace*{0.5cm}
\begin{center}
  ABSTRACT %You need to keep this text here in all capitals. Don't change it.

  [Title]

  [Author]
\end{center}

Lorem ipsum dolor sit amet, consectetur adipisicing elit, sed do eiusmod tempor incididunt ut labore et dolore magna aliqua. Ut enim ad minim veniam, quis nostrud exercitation ullamco laboris nisi ut aliquip ex ea commodo consequat. Duis aute irure dolor in reprehenderit in voluptate velit esse cillum dolore eu fugiat nulla pariatur. Excepteur sint occaecat cupidatat non proident, sunt in culpa qui officia deserunt mollit anim id est laborum.


\hypersetup{pageanchor=true}


%------------------------------------
%Setting up table of contents, list of figures and tables, acknowledgments, and dedication
\frontmatter %This command lets LaTeX know that you want lowercase Roman page numbers in these next sections.
\clearpage\tableofcontents* %LaTeX automatically renders your table of contents using your \chapter and \section commands throughout the whole document. If you don't want something to appear in the table of contents, simply use an asterisk in the command, like \chapter*{} or \section*{}

%A list of graphs and illustrations should go here if you use any.
\clearpage\listoffigures
\clearpage\listoftables

% Acknowledgements
% This is the acknowledgments page of my

\cleartorecto % A memoir-class command for moving the acknowledgments to a recto page, not verso.
%I use an asterisk because I don't want my acknowledgements in the table of contents.
%I use \chapter to make sure that the acknowledgements go on the correct side of the page when you print out the dissertation.
\chapter*{Acknowledgements}
\thispagestyle{plain} %This page should have roman numbers.

Lorem ipsum dolor sit amet, consectetur adipisicing elit, sed do eiusmod tempor incididunt ut labore et dolore magna aliqua. Ut enim ad minim veniam, quis nostrud exercitation ullamco laboris nisi ut aliquip ex ea commodo consequat. Duis aute irure dolor in reprehenderit in voluptate velit esse cillum dolore eu fugiat nulla pariatur. Excepteur sint occaecat cupidatat non proident, sunt in culpa qui officia deserunt mollit anim id est laborum.


% Dedication
% This is the dedication page of my dissertation.

\cleartorecto % A memoir-class command for moving the dedication to a recto page, not verso.
\thispagestyle{plain} % This page should be numbered
\vspace*{2.5 in}
\begin{center}
  [Dedication]
\end{center}


% Preface if you have one


%------------------------------------
%What follows is the main text of your dissertation. You can comment out lines if you want to exclude them from your document for drafts. Everything after \mainmatter will get Arabic numbers centered on the bottom of the page.

\mainmatter

%This uses subdirectories for each part of the dissertation just to keep the files tidy. LaTeX generates a lot of different files for output, and using subdirectories allows you to find your .tex files more easily.
%This is the introduction of the dissertation

\begin{comment} %This environment allows to compile individual chapters with minimal effort.
%This comment environment has to be active for compiling the dissertation file and inactive to compile this individual chapter.
%Just comment out begin and end comment at the beginning and end of the document as well as uncomment the begin and end comment around the chapter title to compile this chapter.

	\documentclass[12pt,letterpaper,oneside]{article}

	%.........................
	%%INCLUDE HERE all packages that you are using in your Dissertation.
	\usepackage{verbatim} % For long comments. Needed if you want to set the documents up so that you can easily compile individual tex files.
	\usepackage{graphicx} %Allows including images
	\usepackage[space]{grffile} %To allow your path to contain spaces (I had everything on my Google Drive)
	\usepackage{booktabs} %For nice lines in tables
	\usepackage{longtable} %For using tables that are longer than one page
	\usepackage{float} %to have "stuff" figures, footnote etc where it should be
	\usepackage{rotating} %For rotating things such as tables (e.g. sidewaystable environment)
	\usepackage{enumitem} % To fix spacing of bullet point lists
	\setlist{noitemsep} %removes space between items
	%\usepackage{afterpage} %might be useful if you run into spacing problems
	%\usepackage[labelfont={bf},font={small}]{caption} %Adjusting table and figure captions the way I prefer them
	\usepackage[hidelinks, plainpages=false, pdfpagelabels]{hyperref} % Use if you want hyperlinks (e.g. in your table of contents). If you don't want them to be blue use the option "hidelinks".
	%hperref sometimes has compatibility issues with other packages, which can often be solved by loading hyperref last. Otherwise more warnings appear about destinations with same identifier.

	\newcommand{\totalN}{10,000 } %An important number that you can use in the text
%.........................

	%Some packages and options that are specific to the individual chapter file
	\graphicspath{{"../Figures//"}} %This has to be here different than in the main dissertation file. Except if you actually use the complete path.
	\usepackage{setspace} %Setting spacing in individual chapter document
	\doublespacing %For double spacing in the individual chapter document
	\usepackage[margin=1.2in]{geometry} %Setting margins in individual chapter document

	\title{Introduction}
	\author{your name}
	\date{Version: \today}

	\begin{document}
	\maketitle
	%\tableofcontents
\end{comment}

%---------------------------------------------------------------
%\begin{comment}
%This comment environment has to be inactive for compiling the dissertation file and active to compile individual chapter.
	\chapter*[ ]{Introduction} %Your introduction isn't a numbered chapter, so we use the asterisk here.
  \addcontentsline{toc}{chapter}{Introduction} %This command puts your introduction in your table of contents even though we have used the asterisk in the \chapter command above.
%\end{comment}

%---------------------------------------------------------------

Here you can write some introductory remarks about your chapter.
I like to give each sentence its own line.

When you need a new paragraph, just skip an extra line.

\section*{New Section}

By using the asterisk to start a new section, I keep the section from appearing in the table of contents.
If you want your sections to be numbered and to appear in the table of contents, remove the asterisk.


\section*{Another section}

Lorem ipsum dolor sit amet, consectetur adipisicing elit, sed do eiusmod tempor incididunt ut labore et dolore magna aliqua. Ut enim ad minim veniam, quis nostrud exercitation ullamco laboris nisi ut aliquip ex ea commodo consequat. Duis aute irure dolor in reprehenderit in voluptate velit esse cillum dolore eu fugiat nulla pariatur. Excepteur sint occaecat cupidatat non proident, sunt in culpa qui officia deserunt mollit anim id est laborum.


%---------------------------------------------------------------
\begin{comment}
%This comment environment has to be active for compiling the dissertation file and inactive to compile individual chapter.
	\bibliographystyle{apalike}
	\bibliography{../Bib_temp}{}

	\end{document}
\end{comment}
%---------------------------------------------------------------

%This is the first chapter of the dissertation

\begin{comment} %This environment allows to compile individual chapters with minimal effort.
%This comment environment has to be active for compiling the dissertation file and inactive to compile this individual chapter.
%Just comment out begin and end comment at the beginning and end of the document as well as uncomment the begin and end comment around the chapter title to compile this chapter.

	\documentclass[12pt,letterpaper,oneside]{article}

	%.........................
	%%INCLUDE HERE all packages that you are using in your Dissertation.
	\usepackage{verbatim} % For long comments. Needed if you want to set the documents up so that you can easily compile individual tex files.
	\usepackage{graphicx} %Allows including images
	\usepackage[space]{grffile} %To allow your path to contain spaces (I had everything on my Google Drive)
	\usepackage{booktabs} %For nice lines in tables
	\usepackage{longtable} %For using tables that are longer than one page
	\usepackage{float} %to have "stuff" figures, footnote etc where it should be
	\usepackage{rotating} %For rotating things such as tables (e.g. sidewaystable environment)
	\usepackage{enumitem} % To fix spacing of bullet point lists
	\setlist{noitemsep} %removes space between items
	%\usepackage{afterpage} %might be useful if you run into spacing problems
	%\usepackage[labelfont={bf},font={small}]{caption} %Adjusting table and figure captions the way I prefer them
	\usepackage[hidelinks, plainpages=false, pdfpagelabels]{hyperref} % Use if you want hyperlinks (e.g. in your table of contents). If you don't want them to be blue use the option "hidelinks".
	%hperref sometimes has compatibility issues with other packages, which can often be solved by loading hyperref last. Otherwise more warnings appear about destinations with same identifier.

	\newcommand{\totalN}{10,000 } %An important number that you can use in the text
%.........................

	%Some packages and options that are specific to the individual chapter file
	\graphicspath{{"../Figures//"}} %This has to be here different than in the main dissertation file. Except if you actually use the complete path.
	\usepackage{setspace} %Setting spacing in individual chapter document
	\doublespacing %For double spacing in the individual chapter document
	\usepackage[margin=1.2in]{geometry} %Setting margins in individual chapter document

	\title{Chapter 1 Title}
	\author{your name}
	\date{Version: \today}

	\begin{document}
	\maketitle
	%\tableofcontents
\end{comment}

%---------------------------------------------------------------
%\begin{comment}
%This comment environment has to be inactive for compiling the dissertation file and active to compile individual chapter.
	\chapter{This is the title of the first chapter}
	\newpage
%\end{comment}

%---------------------------------------------------------------

\section{Introduction}

Here you can write some introductory remarks about your chapter.
Charles McNamara likes to give each sentence its own line.

When you need a new paragraph, just skip an extra line.

The study includes a total of \totalN observations.

I am citing a paper\cite{citkey}.

\section*{New Section}

By using the asterisk to start a new section, I keep the section from appearing in the table of contents.
If you want your sections to be numbered and to appear in the table of contents, remove the asterisk.

Lorem ipsum dolor sit amet, consectetur adipisicing elit, sed do eiusmod tempor incididunt ut labore et dolore magna aliqua. Ut enim ad minim veniam, quis nostrud exercitation ullamco laboris nisi ut aliquip ex ea commodo consequat. Duis aute irure dolor in reprehenderit in voluptate velit esse cillum dolore eu fugiat nulla pariatur. Excepteur sint occaecat cupidatat non proident, sunt in culpa qui officia deserunt mollit anim id est laborum.

\section{Examples}
I am talking about Figure \ref{fig:blue}.

\begin{equation}
	X_{i,t}=\alpha e^{ir_t}
\end{equation}

Lorem ipsum dolor sit amet, consectetur adipisicing elit, sed do eiusmod tempor incididunt ut labore et dolore magna aliqua. Ut enim ad minim veniam, quis nostrud exercitation ullamco laboris nisi ut aliquip ex ea commodo consequat. Duis aute irure dolor in reprehenderit in voluptate velit esse cillum dolore eu fugiat nulla pariatur. Excepteur sint occaecat cupidatat non proident, sunt in culpa qui officia deserunt mollit anim id est laborum.

\begin{figure}
	\center
	\includegraphics[width=0.65\textwidth]{BlueExampleFigure.PDF}
	\caption[Blue Example]{This is the caption of the blue example figure.}
	\label{fig:blue}
\end{figure}

Lorem ipsum dolor sit amet, consectetur adipisicing elit, sed do eiusmod tempor incididunt ut labore et dolore magna aliqua. Ut enim ad minim veniam, quis nostrud exercitation ullamco laboris nisi ut aliquip ex ea commodo consequat. Duis aute irure dolor in reprehenderit in voluptate velit esse cillum dolore eu fugiat nulla pariatur. Excepteur sint occaecat cupidatat non proident, sunt in culpa qui officia deserunt mollit anim id est laborum.

\begin{itemize}
	\item list item 1
	\item list item 2
	\item list item 3
\end{itemize}

Lorem ipsum dolor sit amet, consectetur adipisicing elit, sed do eiusmod tempor incididunt ut labore et dolore magna aliqua. Ut enim ad minim veniam, quis nostrud exercitation ullamco laboris nisi ut aliquip ex ea commodo consequat. Duis aute irure dolor in reprehenderit in voluptate velit esse cillum dolore eu fugiat nulla pariatur. Excepteur sint occaecat cupidatat non proident, sunt in culpa qui officia deserunt mollit anim id est laborum.

Here is a time line
\begin{description}
	\item[1900] Long ago
	\item[1950] A while ago
	\item[2000] Closest to the present.
	\item[2050] This is the future
\end{description}

Lorem ipsum dolor sit amet, consectetur adipisicing elit, sed do eiusmod tempor incididunt ut labore et dolore magna aliqua. Ut enim ad minim veniam, quis nostrud exercitation ullamco laboris nisi ut aliquip ex ea commodo consequat. Duis aute irure dolor in reprehenderit in voluptate velit esse cillum dolore eu fugiat nulla pariatur. Excepteur sint occaecat cupidatat non proident, sunt in culpa qui officia deserunt mollit anim id est laborum.

\begin{sidewaystable}
	\caption[A Table]{This is the table caption of a sideways table}
	\centering
	\begin{tabular}{lll}
		\toprule
		A & B & C \\
		\midrule
		1 & 2 & 3 \\
		3 & 4 & 5 \\
		\bottomrule
	\end{tabular}
	\label{table:abcside}
\end{sidewaystable}

Lorem ipsum dolor sit amet, consectetur adipisicing elit, sed do eiusmod tempor incididunt ut labore et dolore magna aliqua. Ut enim ad minim veniam, quis nostrud exercitation ullamco laboris nisi ut aliquip ex ea commodo consequat. Duis aute irure dolor in reprehenderit in voluptate velit esse cillum dolore eu fugiat nulla pariatur. Excepteur sint occaecat cupidatat non proident, sunt in culpa qui officia deserunt mollit anim id est laborum.

\begin{table}
	\caption[Another Table]{This is the table caption of a regular table}
	\centering
	\begin{tabular}{lll}
		\toprule
		A & B & C \\
		\midrule
		1 & 2 & 3 \\
		3 & 4 & 5 \\
		\bottomrule
	\end{tabular}
	\label{table:abc}
\end{table}


%---------------------------------------------------------------
\begin{comment}
%This comment environment has to be active for compiling the dissertation file and inactive to compile individual chapter.
	\bibliographystyle{apalike}
	\bibliography{../Bib_temp}{}

	\end{document}
\end{comment}
%---------------------------------------------------------------

\begin{appendices}
  \chapter{Appendix to Chapter 1}
  \include{./Chapter1/Appendix1}
\end{appendices}
\include{./Chapter2/Chapter2}
\begin{appendices}
  \chapter{Appendix to Chapter 2}
  \include{./Chapter2/Appendix2}
\end{appendices}
%This is the third chapter of the dissertation

\begin{comment} %This environment allows to compile individual chapters with minimal effort.
%This comment environment has to be active for compiling the dissertation file and inactive to compile this individual chapter.
%Just comment out begin and end comment at the beginning and end of the document as well as uncomment the begin and end comment around the chapter title to compile this chapter.

	\documentclass[12pt,letterpaper,oneside]{article}

	%.........................
	%%INCLUDE HERE all packages that you are using in your Dissertation.
	\usepackage{verbatim} % For long comments. Needed if you want to set the documents up so that you can easily compile individual tex files.
	\usepackage{graphicx} %Allows including images
	\usepackage[space]{grffile} %To allow your path to contain spaces (I had everything on my Google Drive)
	\usepackage{booktabs} %For nice lines in tables
	\usepackage{longtable} %For using tables that are longer than one page
	\usepackage{float} %to have "stuff" figures, footnote etc where it should be
	\usepackage{rotating} %For rotating things such as tables (e.g. sidewaystable environment)
	\usepackage{enumitem} % To fix spacing of bullet point lists
	\usepackage{appendix}
	\setlist{noitemsep} %removes space between items
	%\usepackage{afterpage} %might be useful if you run into spacing problems
	%\usepackage[labelfont={bf},font={small}]{caption} %Adjusting table and figure captions the way I prefer them
	\usepackage[hidelinks, plainpages=false, pdfpagelabels]{hyperref} % Use if you want hyperlinks (e.g. in your table of contents). If you don't want them to be blue use the option "hidelinks".
	%hperref sometimes has compatibility issues with other packages, which can often be solved by loading hyperref last. Otherwise more warnings appear about destinations with same identifier.

	\newcommand{\totalN}{10,000 } %An important number that you can use in the text
%.........................

	%Some packages and options that are specific to the individual chapter file
	\graphicspath{{"../Figures//"}} %This has to be here different than in the main dissertation file. Except if you actually use the complete path.
	\usepackage{setspace} %Setting spacing in individual chapter document
	\doublespacing %For double spacing in the individual chapter document
	\usepackage[margin=1.2in]{geometry} %Setting margins in individual chapter document

	\title{Chapter 3 Title}
	\author{your name}
	\date{Version: \today}

	\begin{document}
	\maketitle
	%\tableofcontents
\end{comment}

%---------------------------------------------------------------
%\begin{comment}
%This comment environment has to be inactive for compiling the dissertation file and active to compile individual chapter.
	\chapter{This is the title of the third chapter}
	\newpage
%\end{comment}

%---------------------------------------------------------------

\section{Introduction}

Here you can write some introductory remarks about your chapter.
I like to give each sentence its own line.

When you need a new paragraph, just skip an extra line.


%---------------------------------------------------------------
\begin{comment}
%This comment environment has to be active for compiling the dissertation file and inactive to compile individual chapter.
	\bibliographystyle{apalike}
	\bibliography{../Bib_temp}{}

	\begin{appendices}
	  \chapter{Appendix to Chapter 3}
	  \include{Appendix3}
	\end{appendices}

	\end{document}
\end{comment}
%---------------------------------------------------------------

\begin{appendices}
  \chapter{Appendix to Chapter 3}
  \include{./Chapter3/Appendix3}
\end{appendices}
\include{./Conclusion/Conclusion}


%------------------------------------
%This final section includes your bibliography.

\backmatter

\SingleSpacing %Start single-spacing text before you start the bibliography. We used \bibitemsep earlier in this document to keep bibliography items separated by one line of blank space, but we need to keep the entries themselves single-spaced.

%Bibliography, again different than Charles McNamara's version
%Your bibliography file is defined as Bib_temp.bib and should be kept in the same folder as this file.
%You probably want to rename it.
\bibliographystyle{apalike}
\bibliography{Bib_temp}

\end{document} %All done! Now you're a doctor.
